% Intended LaTeX compiler: pdflatex
\documentclass[12pt,onecolumn,notitlepage]{revtex4-1}
\usepackage[utf8]{inputenc}
\usepackage[T1]{fontenc}
\usepackage{graphicx}
\usepackage{grffile}
\usepackage{longtable}
\usepackage{wrapfig}
\usepackage{rotating}
\usepackage[normalem]{ulem}
\usepackage{amsmath}
\usepackage{textcomp}
\usepackage{amssymb}
\usepackage{capt-of}
\usepackage{mathpazo,amssymb}
\usepackage[pdftex]{hyperref}
\hypersetup{colorlinks=true}
\begin{document}
\author{A. Scemama, P.-F. Loos}
\date{\today}
\title{ERC PTEROSOR\\\medskip
\large Data Management Plan}
\hypersetup{
 pdfauthor={A. Scemama, P.-F. Loos},
 pdftitle={ERC PTEROSOR},
 pdfkeywords={Software, Notebooks, Publications, Presentations},
 pdfsubject={Org-mode document for \LaTeX{} export},
 pdfcreator={Emacs 26.3 (Org mode 9.1.9)}, 
 pdflang={English}}

\maketitle

\section{Project details}
\label{sec:org2420f0d}

\begin{center}
\begin{tabular}{ll}
\hline
Title                  & PTEROSOR\\
Funder                 & European Research Council, ERC\\
Number                 & 863481\\
Principal Investigator & Pierre-Francois Loos\\
ORCID iD               & 0000-0003-0598-7425\\
Affiliation            & Laboratoire de Chimie et Physique Quantiques\\
                       & (LCPQ) CNRS\\
Data Contact Person    & Anthony Scemama\\
\hline
\end{tabular}
\end{center}

\section{Summary}

Catalysis and solar cell technologies are underpinned by a fundamental
process: that of exciting systems to a higher energy level than the
ground state. Defining an effective method to achieve this that also
provides accurate energies of the excited states is often a
challenge. The EU-funded PTEROSOR project will tackle this fundamental
problem using mathematical techniques. The researchers' novel approach
for measuring the energies of excited states and defining wave
functions in molecular systems will hinge on the use of a general
class of Hamiltonians with parity-time (PT) symmetry. The gateway
between ground and excited states will be provided by exceptional
points which lie at the boundary between broken and unbroken
PT-symmetric regions.


\section{Research outputs}

\begin{enumerate}
\item QuAcK (Software)
\item Quantum Package (Software)
\item Notebooks (Interactive Resource)
\item Textual data (Text)
\end{enumerate}



\section{Dataset summary}

QuAcK is a small quantum chemistry package written in Fortran by the
coordinator of the project. It is mostly used for prototyping.
The size of the archive containing the source code is around 4MiB, and
is composed of Fortran source files, Python and Bash scripts, and
Makefiles.

Quantum Package is an open-source quantum chemistry package for
performing selected configuration interaction calculations with
perturbation theory for molecules and solids.
The project was initiated in 2015 at the LCPQ, and it is now
developed on three sites : Toulouse (LCPQ Toulouse, France), Paris
(LCT Paris, France) and Argonne (USA). Quantum Package is one of
the flagship codes of the TREX European Center of Excellence.
The size of the archive is around 11MiB, and is composed of IRPF90
source files, Python, Bash scripts, Makefiles and standard atomic
basis sets and pseudo-potentials.

All along the project, Mathematica / Jupyter / Org-mode notebooks will
be produced by researchers and students. Each notebook will take a
few megabytes.

PDF files for reports, publications, presentations and posters will
be produced in this project. Each pdf file will take a few megabytes.
Care will be taken to maintain the size of these documents as small
as possible.




\section{FAIR data and resources}

\subsection{Making data findable}

QuAcK is hosted on GitHub (\url{https://github.com/pfloos/quack}), with a
mirror on the Git repository of the LCPQ
(\url{https://git.irsamc.ups-tlse.fr/scemama/quack}). The latest
version was uploaded on Zenodo (\url{doi:10.5281/zenodo.3745928}).

Quantum Package is hosted on GitHub
(\url{https://github.com/QuantumPackage/qp2}), with a mirror on the Git
repository of the LCPQ (\url{https://git.irsamc.ups-tlse.fr/LCPQ/qp2}).
The latest version of the program was presented in a \href{doi:10.1021/acs.jctc.9b00176}{peer-reviewed
article}, and the corresponding preprint was published on \href{https://arxiv.org/abs/1902.08154}{ArXiv} and \href{https://hal.archives-ouvertes.fr/hal-02045595}{HAL}.
The associated source code was uploaded on Zenodo
(\url{doi:10.5281/zenodo.3677565}), and the source code contains a
\texttt{CITATION.cff} file providing metadata in standard YAML format.
Quantum Package has its dedicated website
(\url{https://quantumpackage.github.io/qp2}) providing links to the
GitHub repository, the documentation
(\url{https://quantum-package.readthedocs.io}), and video tutorials
hosted on a YouTube channel
(\url{https://www.youtube.com/channel/UC3a7Yakg9gk36G3HKDIFaYw}). Quantum
Package has also a twitter account (\texttt{@quantum\_package}).

All the notebooks will be versioned in the Git repository of the LCPQ
(\url{https://git.irsamc.ups-tlse.fr}), publicly accessible.
These documents will be archived on Figshare or Zenodo, and the
DOIs will be provided in publications.

The \LaTeX{} source files relative to reports, publications,
presentations and posters will be versioned in the Git repository
of the LCPQ, publicly accessible.

We plan to submit all the source codes involved in this project to
the \href{https://www.softwareheritage.org}{Software Heritage} archive. 

A web site for the project will be created, centralizing all the
links to the archived data. Care will be taken to provide useful
metadata in the HTML headers to help the search engines reference
this web site.

\subsection{Making data openly accessible}

QuAcK is released under the GPL v3 license.
Quantum Package is released under the AGPL v3 license.

Both GitHub repositories are set up to automatically upload on
Zenodo every new release. For each publication requiring a
modification of the source code, a release will be made and the
Zenodo DOI will be cited.

The project doesn't require any part of the codes to be confidential.

\subsection{Making data interoperable}

Standard \texttt{xyz} format is used for atomic coordinates and \texttt{GAMESS/US}
format for atomic basis sets is used. The Basis Set Exchange (BSE)
website (\url{https://www.basissetexchange.org}) provides data in this
format.

Quantum Package is already interfaced with multiple codes of the
community (GAMESS, Molpro, Gaussian, QMCPack, QMC=Chem, \ldots{})

QuAcK operates internally with text files, using the same
conventions as Quantum Package. Hence, the two codes are
compatible and easily interoperable.

\subsection{Increase data reuse}

Along the project, the code will be structured in independent
inter-operating components to make easier the extraction of a
particular feature of the package.

Continuous integration will be set up to guarantee that the package is
functional in the main branch. Developments will be made in a
secondary development branch.

\subsection{Allocation of resources and data security}

The mirroring of the GitHub repository in the institute of the
coordinator provides a backup of both QuAcK and Quantum Package.

The automatic upload of new releases on Zenodo provides secure storage
and long-term preservation of the source code.

We also plan to upload the code in the Software Heritage digital
archive.
\end{document}
